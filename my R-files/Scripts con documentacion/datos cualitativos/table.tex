\documentclass[]{article}
\usepackage{lmodern}
\usepackage{amssymb,amsmath}
\usepackage{ifxetex,ifluatex}
\usepackage{fixltx2e} % provides \textsubscript
\ifnum 0\ifxetex 1\fi\ifluatex 1\fi=0 % if pdftex
  \usepackage[T1]{fontenc}
  \usepackage[utf8]{inputenc}
\else % if luatex or xelatex
  \ifxetex
    \usepackage{mathspec}
  \else
    \usepackage{fontspec}
  \fi
  \defaultfontfeatures{Ligatures=TeX,Scale=MatchLowercase}
\fi
% use upquote if available, for straight quotes in verbatim environments
\IfFileExists{upquote.sty}{\usepackage{upquote}}{}
% use microtype if available
\IfFileExists{microtype.sty}{%
\usepackage{microtype}
\UseMicrotypeSet[protrusion]{basicmath} % disable protrusion for tt fonts
}{}
\usepackage[margin=1in]{geometry}
\usepackage{hyperref}
\hypersetup{unicode=true,
            pdftitle={Untitled},
            pdfauthor={Eric},
            pdfborder={0 0 0},
            breaklinks=true}
\urlstyle{same}  % don't use monospace font for urls
\usepackage{color}
\usepackage{fancyvrb}
\newcommand{\VerbBar}{|}
\newcommand{\VERB}{\Verb[commandchars=\\\{\}]}
\DefineVerbatimEnvironment{Highlighting}{Verbatim}{commandchars=\\\{\}}
% Add ',fontsize=\small' for more characters per line
\usepackage{framed}
\definecolor{shadecolor}{RGB}{248,248,248}
\newenvironment{Shaded}{\begin{snugshade}}{\end{snugshade}}
\newcommand{\AlertTok}[1]{\textcolor[rgb]{0.94,0.16,0.16}{#1}}
\newcommand{\AnnotationTok}[1]{\textcolor[rgb]{0.56,0.35,0.01}{\textbf{\textit{#1}}}}
\newcommand{\AttributeTok}[1]{\textcolor[rgb]{0.77,0.63,0.00}{#1}}
\newcommand{\BaseNTok}[1]{\textcolor[rgb]{0.00,0.00,0.81}{#1}}
\newcommand{\BuiltInTok}[1]{#1}
\newcommand{\CharTok}[1]{\textcolor[rgb]{0.31,0.60,0.02}{#1}}
\newcommand{\CommentTok}[1]{\textcolor[rgb]{0.56,0.35,0.01}{\textit{#1}}}
\newcommand{\CommentVarTok}[1]{\textcolor[rgb]{0.56,0.35,0.01}{\textbf{\textit{#1}}}}
\newcommand{\ConstantTok}[1]{\textcolor[rgb]{0.00,0.00,0.00}{#1}}
\newcommand{\ControlFlowTok}[1]{\textcolor[rgb]{0.13,0.29,0.53}{\textbf{#1}}}
\newcommand{\DataTypeTok}[1]{\textcolor[rgb]{0.13,0.29,0.53}{#1}}
\newcommand{\DecValTok}[1]{\textcolor[rgb]{0.00,0.00,0.81}{#1}}
\newcommand{\DocumentationTok}[1]{\textcolor[rgb]{0.56,0.35,0.01}{\textbf{\textit{#1}}}}
\newcommand{\ErrorTok}[1]{\textcolor[rgb]{0.64,0.00,0.00}{\textbf{#1}}}
\newcommand{\ExtensionTok}[1]{#1}
\newcommand{\FloatTok}[1]{\textcolor[rgb]{0.00,0.00,0.81}{#1}}
\newcommand{\FunctionTok}[1]{\textcolor[rgb]{0.00,0.00,0.00}{#1}}
\newcommand{\ImportTok}[1]{#1}
\newcommand{\InformationTok}[1]{\textcolor[rgb]{0.56,0.35,0.01}{\textbf{\textit{#1}}}}
\newcommand{\KeywordTok}[1]{\textcolor[rgb]{0.13,0.29,0.53}{\textbf{#1}}}
\newcommand{\NormalTok}[1]{#1}
\newcommand{\OperatorTok}[1]{\textcolor[rgb]{0.81,0.36,0.00}{\textbf{#1}}}
\newcommand{\OtherTok}[1]{\textcolor[rgb]{0.56,0.35,0.01}{#1}}
\newcommand{\PreprocessorTok}[1]{\textcolor[rgb]{0.56,0.35,0.01}{\textit{#1}}}
\newcommand{\RegionMarkerTok}[1]{#1}
\newcommand{\SpecialCharTok}[1]{\textcolor[rgb]{0.00,0.00,0.00}{#1}}
\newcommand{\SpecialStringTok}[1]{\textcolor[rgb]{0.31,0.60,0.02}{#1}}
\newcommand{\StringTok}[1]{\textcolor[rgb]{0.31,0.60,0.02}{#1}}
\newcommand{\VariableTok}[1]{\textcolor[rgb]{0.00,0.00,0.00}{#1}}
\newcommand{\VerbatimStringTok}[1]{\textcolor[rgb]{0.31,0.60,0.02}{#1}}
\newcommand{\WarningTok}[1]{\textcolor[rgb]{0.56,0.35,0.01}{\textbf{\textit{#1}}}}
\usepackage{graphicx,grffile}
\makeatletter
\def\maxwidth{\ifdim\Gin@nat@width>\linewidth\linewidth\else\Gin@nat@width\fi}
\def\maxheight{\ifdim\Gin@nat@height>\textheight\textheight\else\Gin@nat@height\fi}
\makeatother
% Scale images if necessary, so that they will not overflow the page
% margins by default, and it is still possible to overwrite the defaults
% using explicit options in \includegraphics[width, height, ...]{}
\setkeys{Gin}{width=\maxwidth,height=\maxheight,keepaspectratio}
\IfFileExists{parskip.sty}{%
\usepackage{parskip}
}{% else
\setlength{\parindent}{0pt}
\setlength{\parskip}{6pt plus 2pt minus 1pt}
}
\setlength{\emergencystretch}{3em}  % prevent overfull lines
\providecommand{\tightlist}{%
  \setlength{\itemsep}{0pt}\setlength{\parskip}{0pt}}
\setcounter{secnumdepth}{0}
% Redefines (sub)paragraphs to behave more like sections
\ifx\paragraph\undefined\else
\let\oldparagraph\paragraph
\renewcommand{\paragraph}[1]{\oldparagraph{#1}\mbox{}}
\fi
\ifx\subparagraph\undefined\else
\let\oldsubparagraph\subparagraph
\renewcommand{\subparagraph}[1]{\oldsubparagraph{#1}\mbox{}}
\fi

%%% Use protect on footnotes to avoid problems with footnotes in titles
\let\rmarkdownfootnote\footnote%
\def\footnote{\protect\rmarkdownfootnote}

%%% Change title format to be more compact
\usepackage{titling}

% Create subtitle command for use in maketitle
\providecommand{\subtitle}[1]{
  \posttitle{
    \begin{center}\large#1\end{center}
    }
}

\setlength{\droptitle}{-2em}

  \title{Untitled}
    \pretitle{\vspace{\droptitle}\centering\huge}
  \posttitle{\par}
    \author{Eric}
    \preauthor{\centering\large\emph}
  \postauthor{\par}
      \predate{\centering\large\emph}
  \postdate{\par}
    \date{7/9/2019}

\usepackage{booktabs}
\usepackage{longtable}
\usepackage{array}
\usepackage{multirow}
\usepackage{wrapfig}
\usepackage{float}
\usepackage{colortbl}
\usepackage{pdflscape}
\usepackage{tabu}
\usepackage{threeparttable}
\usepackage{threeparttablex}
\usepackage[normalem]{ulem}
\usepackage{makecell}
\usepackage{xcolor}

\begin{document}
\maketitle

\hypertarget{estadistica-descriptiva-con-datos-cualitativos}{%
\section{Estadística descriptiva con datos
cualitativos}\label{estadistica-descriptiva-con-datos-cualitativos}}

\hypertarget{funcion-table}{%
\subsection{Función table}\label{funcion-table}}

La función table crea una tabla de frecuencias absolutas del vector o
factor dado La función sample crea una muestra aleatoria de los datos
proporcionados

\begin{Shaded}
\begin{Highlighting}[]
\NormalTok{x =}\StringTok{ }\KeywordTok{sample}\NormalTok{(}\DecValTok{1}\OperatorTok{:}\DecValTok{5}\NormalTok{, }\DataTypeTok{size =} \DecValTok{12}\NormalTok{, }\DataTypeTok{replace =} \OtherTok{TRUE}\NormalTok{) }\CommentTok{# 1:5 son los posibles datos (numeros del 1 al 5), size es el número de muestras que toma, replace indica si se pueden repetir las muestras}
\KeywordTok{table}\NormalTok{(x)}
\end{Highlighting}
\end{Shaded}

\begin{verbatim}
## x
## 1 2 3 4 5 
## 2 3 1 5 1
\end{verbatim}

\begin{Shaded}
\begin{Highlighting}[]
\NormalTok{si_no =}\StringTok{ }\KeywordTok{sample}\NormalTok{(}\KeywordTok{c}\NormalTok{(}\StringTok{"si"}\NormalTok{, }\StringTok{"no"}\NormalTok{), }\DataTypeTok{size =} \DecValTok{12}\NormalTok{, }\DataTypeTok{replace =} \OtherTok{TRUE}\NormalTok{)}
\KeywordTok{table}\NormalTok{(si_no)}
\end{Highlighting}
\end{Shaded}

\begin{verbatim}
## si_no
## no si 
##  7  5
\end{verbatim}

Se puede acceder a los elementos de la tabla por el indice de la columna
o por el nombre de la misma

\begin{Shaded}
\begin{Highlighting}[]
\KeywordTok{table}\NormalTok{(x)[}\DecValTok{4}\NormalTok{]}
\end{Highlighting}
\end{Shaded}

\begin{verbatim}
## 4 
## 5
\end{verbatim}

\begin{Shaded}
\begin{Highlighting}[]
\KeywordTok{table}\NormalTok{(x)[}\StringTok{"4"}\NormalTok{] }\CommentTok{# Si devuelve NA es porque no aparece el valor que se busca en la muestra}
\end{Highlighting}
\end{Shaded}

\begin{verbatim}
## 4 
## 5
\end{verbatim}

\begin{Shaded}
\begin{Highlighting}[]
\KeywordTok{table}\NormalTok{(x)[}\StringTok{"5"}\NormalTok{]}
\end{Highlighting}
\end{Shaded}

\begin{verbatim}
## 5 
## 1
\end{verbatim}

\hypertarget{funcion-prop.table}{%
\subsection{Función prop.table}\label{funcion-prop.table}}

La función prop.table crea la tabla de frecuencias relativas \textbf{de
la tabla de frecuencias absolutas} dada

\begin{Shaded}
\begin{Highlighting}[]
\KeywordTok{prop.table}\NormalTok{(}\KeywordTok{table}\NormalTok{(x))}
\end{Highlighting}
\end{Shaded}

\begin{verbatim}
## x
##          1          2          3          4          5 
## 0.16666667 0.25000000 0.08333333 0.41666667 0.08333333
\end{verbatim}

\begin{Shaded}
\begin{Highlighting}[]
\KeywordTok{prop.table}\NormalTok{(}\KeywordTok{table}\NormalTok{(si_no))}
\end{Highlighting}
\end{Shaded}

\begin{verbatim}
## si_no
##        no        si 
## 0.5833333 0.4166667
\end{verbatim}

\hypertarget{tablas-de-frecuencia-bidimensionales}{%
\subsection{Tablas de frecuencia
bidimensionales}\label{tablas-de-frecuencia-bidimensionales}}

La función table permite crear tablas de frecuencias de más de una
dimensión

\begin{Shaded}
\begin{Highlighting}[]
\NormalTok{Genero =}\StringTok{ }\KeywordTok{sample}\NormalTok{(}\KeywordTok{c}\NormalTok{(}\StringTok{"H"}\NormalTok{, }\StringTok{"M"}\NormalTok{), }\DataTypeTok{size =} \KeywordTok{length}\NormalTok{(si_no), }\DataTypeTok{replace =} \OtherTok{TRUE}\NormalTok{)}
\KeywordTok{table}\NormalTok{(si_no, Genero)}
\end{Highlighting}
\end{Shaded}

\begin{verbatim}
##      Genero
## si_no H M
##    no 5 2
##    si 3 2
\end{verbatim}

En las tablas de frecuencia bidimensionales se pueden crear 3 tablas de
frecuencias relativas distintas

\begin{Shaded}
\begin{Highlighting}[]
\KeywordTok{prop.table}\NormalTok{(}\KeywordTok{table}\NormalTok{(si_no, Genero)) }\CommentTok{# Frecuencia relativa global (respecto al total de la tabla)}
\end{Highlighting}
\end{Shaded}

\begin{verbatim}
##      Genero
## si_no         H         M
##    no 0.4166667 0.1666667
##    si 0.2500000 0.1666667
\end{verbatim}

\begin{Shaded}
\begin{Highlighting}[]
\KeywordTok{prop.table}\NormalTok{(}\KeywordTok{table}\NormalTok{(si_no, Genero), }\DataTypeTok{margin =} \DecValTok{1}\NormalTok{) }\CommentTok{# Frecuencia relativa marginal (respecto al total de la fila)}
\end{Highlighting}
\end{Shaded}

\begin{verbatim}
##      Genero
## si_no         H         M
##    no 0.7142857 0.2857143
##    si 0.6000000 0.4000000
\end{verbatim}

\begin{Shaded}
\begin{Highlighting}[]
\KeywordTok{prop.table}\NormalTok{(}\KeywordTok{table}\NormalTok{(si_no, Genero), }\DataTypeTok{margin =} \DecValTok{2}\NormalTok{) }\CommentTok{# Frecuencia relativa marginal (respecto al total de la columna)}
\end{Highlighting}
\end{Shaded}

\begin{verbatim}
##      Genero
## si_no     H     M
##    no 0.625 0.500
##    si 0.375 0.500
\end{verbatim}

gmodels es una librería que, entre otras cosas, permite generar dichas
tablas y mostrarlas en una sola

\begin{Shaded}
\begin{Highlighting}[]
\KeywordTok{library}\NormalTok{(gmodels)}
\KeywordTok{CrossTable}\NormalTok{(Genero, si_no, }\DataTypeTok{prop.chisq =} \OtherTok{TRUE}\NormalTok{)}
\end{Highlighting}
\end{Shaded}

\begin{verbatim}
## 
##  
##    Cell Contents
## |-------------------------|
## |                       N |
## | Chi-square contribution |
## |           N / Row Total |
## |           N / Col Total |
## |         N / Table Total |
## |-------------------------|
## 
##  
## Total Observations in Table:  12 
## 
##  
##              | si_no 
##       Genero |        no |        si | Row Total | 
## -------------|-----------|-----------|-----------|
##            H |         5 |         3 |         8 | 
##              |     0.024 |     0.033 |           | 
##              |     0.625 |     0.375 |     0.667 | 
##              |     0.714 |     0.600 |           | 
##              |     0.417 |     0.250 |           | 
## -------------|-----------|-----------|-----------|
##            M |         2 |         2 |         4 | 
##              |     0.048 |     0.067 |           | 
##              |     0.500 |     0.500 |     0.333 | 
##              |     0.286 |     0.400 |           | 
##              |     0.167 |     0.167 |           | 
## -------------|-----------|-----------|-----------|
## Column Total |         7 |         5 |        12 | 
##              |     0.583 |     0.417 |           | 
## -------------|-----------|-----------|-----------|
## 
## 
\end{verbatim}

\hypertarget{tabla-de-frecuencia-multidimensional}{%
\subsection{Tabla de frecuencia
multidimensional}\label{tabla-de-frecuencia-multidimensional}}

\begin{Shaded}
\begin{Highlighting}[]
\NormalTok{lugar =}\StringTok{ }\KeywordTok{sample}\NormalTok{(}\KeywordTok{c}\NormalTok{(}\StringTok{"Alicante"}\NormalTok{, }\StringTok{"Valencia"}\NormalTok{, }\StringTok{"Barcelona"}\NormalTok{, }\StringTok{"Madrid"}\NormalTok{, }\StringTok{"Girona"}\NormalTok{), }\DataTypeTok{size =} \KeywordTok{length}\NormalTok{(si_no), }\DataTypeTok{replace =} \OtherTok{TRUE}\NormalTok{)}
\NormalTok{tabla <-}\StringTok{ }\KeywordTok{table}\NormalTok{(si_no, Genero, lugar)}
\KeywordTok{prop.table}\NormalTok{(}\KeywordTok{table}\NormalTok{(si_no, Genero, lugar)) }\CommentTok{# La tabla de frecuencias relativas es preferible hacerla de esta forma}
\end{Highlighting}
\end{Shaded}

\begin{verbatim}
## , , lugar = Alicante
## 
##      Genero
## si_no          H          M
##    no 0.00000000 0.00000000
##    si 0.08333333 0.00000000
## 
## , , lugar = Barcelona
## 
##      Genero
## si_no          H          M
##    no 0.25000000 0.00000000
##    si 0.08333333 0.08333333
## 
## , , lugar = Girona
## 
##      Genero
## si_no          H          M
##    no 0.08333333 0.00000000
##    si 0.00000000 0.08333333
## 
## , , lugar = Madrid
## 
##      Genero
## si_no          H          M
##    no 0.00000000 0.16666667
##    si 0.08333333 0.00000000
## 
## , , lugar = Valencia
## 
##      Genero
## si_no          H          M
##    no 0.08333333 0.00000000
##    si 0.00000000 0.00000000
\end{verbatim}

\begin{Shaded}
\begin{Highlighting}[]
\KeywordTok{prop.table}\NormalTok{(}\KeywordTok{table}\NormalTok{(si_no, Genero, lugar), }\DataTypeTok{margin =} \DecValTok{1}\NormalTok{)}
\end{Highlighting}
\end{Shaded}

\begin{verbatim}
## , , lugar = Alicante
## 
##      Genero
## si_no         H         M
##    no 0.0000000 0.0000000
##    si 0.2000000 0.0000000
## 
## , , lugar = Barcelona
## 
##      Genero
## si_no         H         M
##    no 0.4285714 0.0000000
##    si 0.2000000 0.2000000
## 
## , , lugar = Girona
## 
##      Genero
## si_no         H         M
##    no 0.1428571 0.0000000
##    si 0.0000000 0.2000000
## 
## , , lugar = Madrid
## 
##      Genero
## si_no         H         M
##    no 0.0000000 0.2857143
##    si 0.2000000 0.0000000
## 
## , , lugar = Valencia
## 
##      Genero
## si_no         H         M
##    no 0.1428571 0.0000000
##    si 0.0000000 0.0000000
\end{verbatim}

\begin{Shaded}
\begin{Highlighting}[]
\KeywordTok{prop.table}\NormalTok{(}\KeywordTok{table}\NormalTok{(si_no, Genero, lugar), }\DataTypeTok{margin =} \DecValTok{2}\NormalTok{)}
\end{Highlighting}
\end{Shaded}

\begin{verbatim}
## , , lugar = Alicante
## 
##      Genero
## si_no     H     M
##    no 0.000 0.000
##    si 0.125 0.000
## 
## , , lugar = Barcelona
## 
##      Genero
## si_no     H     M
##    no 0.375 0.000
##    si 0.125 0.250
## 
## , , lugar = Girona
## 
##      Genero
## si_no     H     M
##    no 0.125 0.000
##    si 0.000 0.250
## 
## , , lugar = Madrid
## 
##      Genero
## si_no     H     M
##    no 0.000 0.500
##    si 0.125 0.000
## 
## , , lugar = Valencia
## 
##      Genero
## si_no     H     M
##    no 0.125 0.000
##    si 0.000 0.000
\end{verbatim}

\begin{Shaded}
\begin{Highlighting}[]
\KeywordTok{prop.table}\NormalTok{(}\KeywordTok{table}\NormalTok{(si_no, Genero, lugar), }\DataTypeTok{margin =} \DecValTok{3}\NormalTok{)}
\end{Highlighting}
\end{Shaded}

\begin{verbatim}
## , , lugar = Alicante
## 
##      Genero
## si_no         H         M
##    no 0.0000000 0.0000000
##    si 1.0000000 0.0000000
## 
## , , lugar = Barcelona
## 
##      Genero
## si_no         H         M
##    no 0.6000000 0.0000000
##    si 0.2000000 0.2000000
## 
## , , lugar = Girona
## 
##      Genero
## si_no         H         M
##    no 0.5000000 0.0000000
##    si 0.0000000 0.5000000
## 
## , , lugar = Madrid
## 
##      Genero
## si_no         H         M
##    no 0.0000000 0.6666667
##    si 0.3333333 0.0000000
## 
## , , lugar = Valencia
## 
##      Genero
## si_no         H         M
##    no 1.0000000 0.0000000
##    si 0.0000000 0.0000000
\end{verbatim}

\begin{Shaded}
\begin{Highlighting}[]
\KeywordTok{prop.table}\NormalTok{(}\KeywordTok{table}\NormalTok{(si_no, Genero, lugar), }\DataTypeTok{margin =} \KeywordTok{c}\NormalTok{(}\DecValTok{1}\NormalTok{,}\DecValTok{2}\NormalTok{))}
\end{Highlighting}
\end{Shaded}

\begin{verbatim}
## , , lugar = Alicante
## 
##      Genero
## si_no         H         M
##    no 0.0000000 0.0000000
##    si 0.3333333 0.0000000
## 
## , , lugar = Barcelona
## 
##      Genero
## si_no         H         M
##    no 0.6000000 0.0000000
##    si 0.3333333 0.5000000
## 
## , , lugar = Girona
## 
##      Genero
## si_no         H         M
##    no 0.2000000 0.0000000
##    si 0.0000000 0.5000000
## 
## , , lugar = Madrid
## 
##      Genero
## si_no         H         M
##    no 0.0000000 1.0000000
##    si 0.3333333 0.0000000
## 
## , , lugar = Valencia
## 
##      Genero
## si_no         H         M
##    no 0.2000000 0.0000000
##    si 0.0000000 0.0000000
\end{verbatim}

\begin{Shaded}
\begin{Highlighting}[]
\KeywordTok{prop.table}\NormalTok{(}\KeywordTok{table}\NormalTok{(si_no, Genero, lugar), }\DataTypeTok{margin =} \KeywordTok{c}\NormalTok{(}\DecValTok{1}\NormalTok{,}\DecValTok{3}\NormalTok{))}
\end{Highlighting}
\end{Shaded}

\begin{verbatim}
## , , lugar = Alicante
## 
##      Genero
## si_no   H   M
##    no        
##    si 1.0 0.0
## 
## , , lugar = Barcelona
## 
##      Genero
## si_no   H   M
##    no 1.0 0.0
##    si 0.5 0.5
## 
## , , lugar = Girona
## 
##      Genero
## si_no   H   M
##    no 1.0 0.0
##    si 0.0 1.0
## 
## , , lugar = Madrid
## 
##      Genero
## si_no   H   M
##    no 0.0 1.0
##    si 1.0 0.0
## 
## , , lugar = Valencia
## 
##      Genero
## si_no   H   M
##    no 1.0 0.0
##    si
\end{verbatim}

\begin{Shaded}
\begin{Highlighting}[]
\KeywordTok{prop.table}\NormalTok{(}\KeywordTok{table}\NormalTok{(si_no, Genero, lugar), }\DataTypeTok{margin =} \KeywordTok{c}\NormalTok{(}\DecValTok{2}\NormalTok{,}\DecValTok{3}\NormalTok{))}
\end{Highlighting}
\end{Shaded}

\begin{verbatim}
## , , lugar = Alicante
## 
##      Genero
## si_no    H M
##    no 0.00  
##    si 1.00  
## 
## , , lugar = Barcelona
## 
##      Genero
## si_no    H    M
##    no 0.75 0.00
##    si 0.25 1.00
## 
## , , lugar = Girona
## 
##      Genero
## si_no    H    M
##    no 1.00 0.00
##    si 0.00 1.00
## 
## , , lugar = Madrid
## 
##      Genero
## si_no    H    M
##    no 0.00 1.00
##    si 1.00 0.00
## 
## , , lugar = Valencia
## 
##      Genero
## si_no    H M
##    no 1.00  
##    si 0.00
\end{verbatim}

\begin{Shaded}
\begin{Highlighting}[]
\KeywordTok{ftable}\NormalTok{(Genero, si_no, lugar) }\CommentTok{# flat table o tabla plana (no divide la tabla en x tablas bidimensionales)}
\end{Highlighting}
\end{Shaded}

\begin{verbatim}
##              lugar Alicante Barcelona Girona Madrid Valencia
## Genero si_no                                                
## H      no                 0         3      1      0        1
##        si                 1         1      0      1        0
## M      no                 0         0      0      2        0
##        si                 0         1      1      0        0
\end{verbatim}

\hypertarget{permutacion-de-variables}{%
\subsection{Permutación de variables}\label{permutacion-de-variables}}

\begin{Shaded}
\begin{Highlighting}[]
\KeywordTok{aperm}\NormalTok{(tabla, }\DataTypeTok{perm =} \KeywordTok{c}\NormalTok{(}\StringTok{"Genero"}\NormalTok{, }\StringTok{"lugar"}\NormalTok{, }\StringTok{"si_no"}\NormalTok{))}
\end{Highlighting}
\end{Shaded}

\begin{verbatim}
## , , si_no = no
## 
##       lugar
## Genero Alicante Barcelona Girona Madrid Valencia
##      H        0         3      1      0        1
##      M        0         0      0      2        0
## 
## , , si_no = si
## 
##       lugar
## Genero Alicante Barcelona Girona Madrid Valencia
##      H        1         1      0      1        0
##      M        0         1      1      0        0
\end{verbatim}

\hypertarget{funcion-kable}{%
\subsection{Función kable}\label{funcion-kable}}

\begin{Shaded}
\begin{Highlighting}[]
\KeywordTok{library}\NormalTok{(kableExtra)}
\KeywordTok{kable}\NormalTok{(tabla) }\CommentTok{# Muestra las frecuencias en filas}
\end{Highlighting}
\end{Shaded}

\begin{tabular}{l|l|l|r}
\hline
si\_no & Genero & lugar & Freq\\
\hline
no & H & Alicante & 0\\
\hline
si & H & Alicante & 1\\
\hline
no & M & Alicante & 0\\
\hline
si & M & Alicante & 0\\
\hline
no & H & Barcelona & 3\\
\hline
si & H & Barcelona & 1\\
\hline
no & M & Barcelona & 0\\
\hline
si & M & Barcelona & 1\\
\hline
no & H & Girona & 1\\
\hline
si & H & Girona & 0\\
\hline
no & M & Girona & 0\\
\hline
si & M & Girona & 1\\
\hline
no & H & Madrid & 0\\
\hline
si & H & Madrid & 1\\
\hline
no & M & Madrid & 2\\
\hline
si & M & Madrid & 0\\
\hline
no & H & Valencia & 1\\
\hline
si & H & Valencia & 0\\
\hline
no & M & Valencia & 0\\
\hline
si & M & Valencia & 0\\
\hline
\end{tabular}


\end{document}
